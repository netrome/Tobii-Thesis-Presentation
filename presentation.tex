\documentclass{beamer}

\usepackage{amsfonts}
\usepackage{amsmath}
\usepackage{siunitx}
\usepackage{graphicx}
\usepackage{subcaption}
\usepackage[acronym]{glossaries}
\makeglossaries
\newacronym{gan}{GAN}{Generative Adversarial Network}
\newacronym{gans}{GANs}{Generative Adversarial Networks}
\newacronym{vae}{VAE}{Variational Autoencoder}
\newacronym{vaes}{VAEs}{Variational Autoencoders}
\newacronym{mse}{MSE}{Mean Squared Error}
\newacronym{aevb}{AEVB}{Auto Encoding Variational Bayes}
\newacronym{cnn}{CNN}{Convolutional Neural Network}
\newacronym{cnns}{CNNs}{Convolutional Neural Networks}
\newacronym{is}{IS}{Inception Score}
\newacronym{fid}{FID}{Fréchet Inception Distance}
\newacronym{wgan}{WGAN}{Wasserstein GAN}
\newacronym{aegan}{AEGAN}{Autoencoding GAN}
\newacronym{roi}{ROI}{Region Of Interest}
\newacronym{vaegan}{VAEGAN}{Variational Autoencoder + Generative Adversarial Network}
  
\usetheme{Tobii}

\title{Data Augmentation with Deep Generative Models}
\author{Mårten Nilsson}
\institute{Tobii}
\date{2018}

\begin{document}

% Tobii presentation of Mårtens GAN thesis project

% Sample frames --------------------------------------------------
\frame{\titlepage}

%\begin{sectionframe}
%    \frametitle{Frame title}
%    Section frame
%\end{sectionframe}
%
%\begin{messageframe}
%    \frametitle{Frame title}
%    message frame
%\end{messageframe}
%
%\messageslide{Message slide}
%
%\statementslide{Statement slide}{images/banana.png}
%\statementslideB{Statement slide B}{images/banana.png}
%
%\plainslide{Slide title}{Plain slide}
%
%\textandpicslideA{Title}{Text and pic slide A}{images/banana.png}
%\textandpicslideB{Title}{Text and pic slide B}{images/banana.png}
%
% Structure of presentation ------
% Agenda and subject presentation
% Establish importance of topic (why generative models?, what are they useful for?, why are we interested in them?)
% Establish problem formulation for generative models, includning some pitfalls (ill posed questions etc).
% Provide some backround knowledge
% Present the tested methods and results
% Interpret results
% Final remarks and conclusions

\plainslide{Outline}{
    \begin{itemize}
        \item Generative Modelling
        \item Deep Generative Models: GANs and VAEs
        \item Data augmentation with Deep Generative Models  % Talk about the current research
        \item Experiments and results
        \item Concluding remarks
    \end{itemize}
}

\messageslide{Generative Modelling}

\begin{messageframe}
    \frametitle{What is a generative model?}

    \begin{itemize}
            \item A model capable of producing data similar to what the model have been trained on
            \item Different from discriminative models
            \item Example: Gaussian Mixture Models
    \end{itemize}

\end{messageframe}

\begin{messageframe}
    \frametitle{Why do we care about generative modelling?}
        \textit{"What I cannot create I do not understand"} 

        - Richard Feynman
    \begin{itemize}
            \item Compression
            \item Augmentation
            \item Anonymization (GDPR)
            \item Domain adaptation

            - Both Apple and Google utilize it for gaze estimation
    \end{itemize}
\end{messageframe}

\messageslide{Deep Generative Models}

\begin{messageframe}
    \frametitle{\acrfull{vaes}}
    \begin{itemize}
            \item Consists of an encoder and a decoder
            \item These can be convolutional neural networks
            \item The encoder is used for modelling $P(latent | data)$
            \item The decoder is used for modelling $P(data | latent)$
            \item Model parameters learned through variational inference
            \item Trained with two objectives
            \begin{itemize}
                \item $P(latent | data)$ should be similar to a specific prior distribution $P(latent)$
                \item $P(data | latent)$ where $latent$ is sampled from $P(latent | data)$ should give a high likelihood to the conditioned data point
            \end{itemize}
    \end{itemize}
\end{messageframe}

\begin{messageframe}
    \frametitle{\acrfull{gans}}

    \begin{itemize}
            \item Consists of a generator and a discriminator
            \item These can be convolutional neural networks
    \end{itemize}
\end{messageframe}

\begin{messageframe}
    \frametitle{\acrshort{gans} in practice}
    List properties and approaches here.
\end{messageframe}

\messageslide{Data Augmentation}

\begin{messageframe}
    \frametitle{The question}
    \textit{"Can deep generative models be applied to generate synthetic data sets that can be used to boost the performance of existing discriminative models?"}
    
    \vspace{1.5cm}
    
    In other words:

    "Can we generate \textbf{useful} data?"
\end{messageframe}

\begin{messageframe}
    \frametitle{Measuring the usability of the data}
    Train classifier separately on both generated and real data.
\end{messageframe}

\begin{messageframe}
    \frametitle{Generating labeled data}
    No special cases needed - there is no difference between labeled and unlabeled data.
\end{messageframe}

\messageslide{Experiments and Results}

\begin{messageframe}
    \frametitle{Tested models}
    WGAN, VAE, AEGAN (Progressive GAN).
\end{messageframe}

\statementslideB{WGAN}{images/generated/wasserstein_fake_6x4.png}

\statementslideB{VAE}{images/generated/vae_fake_6x4.png}

\statementslideB{AEGAN}{images/generated/aegan_fake_6x4.png}

\plainslide{Results: synthetic data}{

    \begin{table}[t]
        \centering
        \begin{subtable}{0.9\textwidth}
            \begin{tabular}{|l|lll|}
                \cline{2-4}
                \multicolumn{1}{c|}{} & \multicolumn{3}{c|}{\textbf{Jaccard distance}} \\ \hline
                \textbf{Training data} & \acrshort{wgan} & VAE & \acrshort{aegan} \\ \hline
                Fake & \num{0.8132} & \num{0.159} & \textbf{\num{0.148}} \\
                Fake + real & \num{0.1219} & \textbf{\num{0.0944}} & \num{0.1032} \\
                \hline
                Real & \multicolumn{3}{c|}{\num{0.09607}} \\
                \hline
            \end{tabular}
            \caption{Tested on the real test data.}
            \label{subtab:on_real}
        \end{subtable}
        \begin{subtable}{0.9\textwidth}
            \begin{tabular}{|l|lll|}
                \cline{2-4}
                \multicolumn{1}{c|}{} & \multicolumn{3}{c|}{\textbf{Jaccard distance}} \\ \hline
                \textbf{Training data} & \acrshort{wgan} & VAE & \acrshort{aegan} \\ \hline
                Fake & \textbf{\num{4.82e-5}} & \num{0.0133} & \num{0.0491} \\ 
                Real & \num{0.1855} & \textbf{\num{0.0264}} & \num{0.1225} \\ 
                \hline
            \end{tabular}
            \caption{Test data generated by the generative models.}
            \label{subtab:on_fake}
        \end{subtable}
    \end{table}
}

\plainslide{Results: DeepGaze data}{
    \begin{table}[t]
        \centering
        \begin{subtable}{0.9\textwidth}
            \begin{tabular}{|l|lll|}
                \cline{2-4}
                \multicolumn{1}{c|}{} & \multicolumn{3}{c|}{\textbf{Jaccard distance}} \\ \hline
                \textbf{Training data} & \acrshort{wgan} & VAE & \acrshort{aegan} \\ \hline
                Fake & \num{0.9966} & \num{0.3445} & \textbf{\num{0.2084}} \\
                Fake + real & \num{0.1215} & \textbf{\num{0.1151}} & \num{0.1221} \\
                \hline
                Real & \multicolumn{3}{c|}{\num{0.11456}} \\
                \hline
            \end{tabular}
            \caption{Tested on the real test data.}
        \end{subtable}
        \begin{subtable}{0.9\textwidth}
            \begin{tabular}{|l|lll|}
                \cline{2-4}
                \multicolumn{1}{c|}{} & \multicolumn{3}{c|}{\textbf{Jaccard distance}} \\ \hline
                \textbf{Training data} & \acrshort{wgan} & VAE & \acrshort{aegan} \\ \hline
                Fake & \textbf{\num{8.795e-6}} & \num{0.0284} & \num{0.0457} \\ 
                Real & \num{0.8798} & \num{0.1008} & \textbf{\num{0.0786}} \\ 
                \hline
            \end{tabular}
            \caption{Test data generated by the generative models.}
        \end{subtable}
    \end{table}
}

\plainslide{Results: toy experiments}{
    \begin{table}[t]
        \centering
        \label{tab:toy_experiments}
        \begin{subtable}{0.9\textwidth}
            \scalebox{0.7}{
                \begin{tabular}{|l|ll|ll|ll|ll|}
                    \cline{2-9}
                    \multicolumn{1}{c|}{ } & \multicolumn{2}{c|}{Baseline} & \multicolumn{2}{c|}{WGAN} & \multicolumn{2}{c|}{VAE} & \multicolumn{2}{c|}{AEGAN} \\ \cline{2-9}
                    \multicolumn{1}{c|}{} & Mean & Std & Mean & Std & Mean & Std & Mean & Std \\ \hline
                    Error rate ($\%$) & \num{32.00} & \num{3.0792} & \num{33.33333} & \num{3.399} & \num{30.666} & \num{4.38178} & \textbf{\num{28.80}} & \num{2.0221} \\
                    \hline
                \end{tabular}
            }
            \caption{Iris}
        \end{subtable}
        \begin{subtable}{0.9\textwidth}
            \scalebox{0.7}{
                \begin{tabular}{|l|lll|l|l|}
                    \cline{2-6}
                    \multicolumn{1}{c|}{} & Baseline 100 & Baseline 200 & Baseline 400 & VAE & AEGAN \\ \hline
                    Error rate ($\%$) & \num{0.90} & \num{1.02} & \num{0.87} & \num{1.33} & \textbf{\num{0.86}} \\ \hline
                \end{tabular}
            }
            \caption{MNIST}
        \end{subtable}
    \end{table}
}


\messageslide{Concluding remarks}

\statementslide{Questions?}{images/banana.png}

\statementslide{Thank you}{images/banana.png}

\end{document}
